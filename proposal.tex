\documentclass[a4paper, 11pt]{article}
\usepackage[utf8]{inputenc}
\usepackage{graphicx}
\usepackage{url}
\usepackage{color}

\title{LabSpace: A Collaborative Space where Knowledge and Experience is Shared}
\author{George E. Kallergis\\geokal@kth.se}
\date{\today{}}

\begin{document}

\maketitle

\begin{figure}[h!]
  \begin{center}
    \includegraphics[width=\textwidth,height=\textheight,keepaspectratio]{imagery/logo.png}
    \label{fig:dneaf}
  \end{center}
\end{figure}

\textit{This document describes the operational model and benefits of LabSpace, a collaborative laboratory space where students and enthusiasts can exchange knowledge and experiences on a wide variety of areas including, but not limited to electronics, programming and art.}

\newpage

% Introduction
\section*{Introduction}
LabSpace is a specially engineered maker-space where students and other enthusiasts can meet and exchange knowledge and experiences around different engineering related matters as well as non-engineering ones (i.e. art). The main vision of LabSpace is to promote the exchange of ideas between interdisciplinary teams and give birth to innovative projects that combine knowledge from a multitude of different areas. Knowledge exchange is also encouraged inside of LabSpace allowing for members to enhance their skills and also get a chance to help (or get helped by) others in order to overcome possible learning difficulties they might be experiencing.

% Benefits
\section{Benefits}
The following two sections discuss in more detail how the implementation of a LabSpace can greatly benefit a university and its students (as LabSpace members). In other words the proposed value of a LabSpace is shown and analyzed from the perspective of different stakeholders.

\subsection{Implementing University}
\textit{Why is it a good idea for a university to have a LabSpace? How will it benefit from it? What will the university have to do in order to implement a LabSpace? Sponsors? Innovation? Benefits for the international image of the university? Cross university cooperations (i.e. Konstfack)?}

\subsection{LabSpace Members}
\textit{How can students benefit from LabSpace? Learning from others that already have the experience? How can that ease their learning curve? Socialize? Network? Share resources and get access to things they couldn't otherwise have access to? Opportunity for startups? Opportunity to teach others? Focus on industry important matters? Benefits on future employment possibilities? Enrich their CV with cool projects?}


% Operation
\section{Operation}

\subsection{LabSpace Business Model}
\textit{How will LabSpace be self sustainable? Which are going to be it's partners and the key activities that need to be undertaken in order for it to operate properly? What are the costs (if any) and how are they going to be covered (the less the financial impact on the university the better)?}

\subsection{Space Layout}
\textit{This is specific to the Electrum lab. It consists of a top view of the lab where the placement of available resources (i.e. teables, PCs, other equipment) is shown. The goal is  to allow for  easy concurrent access for the maximum number of students.}

\subsection{Equipment}
\textit{The minimum required equipment for operation will be discussed here and a list specifically for Electrum can be created.}

\subsection{Space Management}
\textit{The "fair use" policy will be defined here to ensure that the space can be used by anyone, but also specific resources can be bound for future use (e.g. breadboards and components to avoid assembly and disassembly of the circuit on every visit). Access to the building will be discussed here.}

\end{document}

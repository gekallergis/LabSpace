\documentclass[a4paper, 11pt]{article}
\usepackage[utf8]{inputenc}
\usepackage{graphicx}
\usepackage{url}
\usepackage{hyperref}
\usepackage{color}
\usepackage{enumitem}

\title{LabSpace Stakeholder Analysis: Enthusiasts}
\author{George E. Kallergis\\geokal@kth.se}
\date{\today{}}

\begin{document}

\maketitle

\begin{figure}[h!]
  \begin{center}
    \includegraphics[width=\textwidth,height=\textheight,keepaspectratio]{imagery/logo.png}
    \label{fig:dneaf}
  \end{center}
\end{figure}

\textit{This study analyzes the benefits of LabSpace from the perspective of the different stakeholders involved.}

\newpage

\section{Introduction}

\subsection{Background}

\subsection{Problem}

\subsection{Purpose}

\subsection{Goal(s)}

\section{Method}

\subsection{Data Collection}

\subsection{Data Analysis}

\section{Results}

\subsection{Discussion}

\section{Conclusions \& Future Work}

% Questions to Ask!
%Why is it a good idea for a university to have a LabSpace? How will it benefit from it? What will the university have to do in order to implement a LabSpace? Sponsors? Innovation? Benefits for the international image of the university? Cross university cooperations (i.e. Konstfack)?How can students benefit from LabSpace? Learning from others that already have the experience? How can that ease their learning curve? Socialize? Network? Share resources and get access to things they couldn't otherwise have access to? Opportunity for startups? Opportunity to teach others? Focus on industry important matters? Benefits on future employment possibilities? Enrich their CV with cool projects?

\newpage

\begin{thebibliography}{9}
    \bibitem{whatsamakerspace} \emph{What’s a Makerspace?} (n.d.) [Online]. Available: \\ \href{http://makerspace.com/home-page}{http://makerspace.com/home-page} (Accessed: February 3\textsuperscript{rd}, 2014).
\end{thebibliography}

\end{document}
